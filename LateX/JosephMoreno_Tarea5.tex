\documentclass{article}
\usepackage[utf8]{inputenc}
\usepackage{amssymb}
\usepackage{amsmath}
\title{Homework 5 and exercises}
\author{Joseph Edrei Moreno Cruz~\\ 
F\'isica Computacional 1} 
\date{October 2, 2017}
\begin{document}
\maketitle
\begin{center}
\textbf{Ejercicios Diapositiva 8}
\end{center} \par
\begin{displaymath}
\cos^2{\theta} + \sin^2{\theta} = 1
\end{displaymath}
$$\sqrt{2}\approx{1.414}{}\qquad{}\sqrt[3]{2}\approx{1.260}$$

\begin{displaymath}
e^{\pi{}i} = 1
\end{displaymath}
\begin{displaymath}
\frac{\partial^2 f}{\partial x {} \partial y}
\end{displaymath}
\begin{displaymath}
F_{n} = F_{n-1}+F_{n-2}, 
\qquad
n\geq{0}
\end{displaymath}
\begin{displaymath}
A = B \qquad \text{if and only if} \qquad A\subseteq{B} \quad \text{and} \quad A\supseteq{B}
\end{displaymath}
\begin{center}
\textbf{Ejercicios Diapositiva 10}
\end{center} \par
\begin{text}

\end{text}
\begin{align*}
x &= r\cos{\phi}\sin{\theta} \\
y &= r\sin{\phi}\sin{\theta} \\
z &= r\cos{\theta}
\end{align*} \par
\begin{align*}
x+2y-3z &= -11 \\
y \; \; +z &= \; \; \; 11 \\
z &= \; \; \; 21
\end{align*} \par
\begin{center}
$F_{2}^{2}$ and $F{}_{2}^{2}$. \par
$x_{1}^{y}$, $x^{y}_{1}$, and $x^{y_{1}}$.\par

\end{center}
\begin{equation*}
\mathnormal{henry} = 1.113 \times 10^{-12}sec^{2}/cm
\end{equation*}
\\
The equation
$$ax^{2} + bx + c$$
has as solution
$$x_{12} = \frac{-b\pm\sqrt{b^{2} - 4ac}}{2a}$$ \\
\begin{equation}
\epsilon > 0
\end{equation}
From condition (1) follows\ldots \\
\begin{center}
\textbf{Ejercicios Diapositiva Parte 2}
\end{center}
$\textbf{A} = \begin{pmatrix}
a+b+c & uv\\
a+b & u+v
\end{pmatrix}\
\begin{vmatrix}
30 & 7\\
3 & 17
\end{vmatrix}$
\qquad
\qquad
\qquad
$A = \begin{pmatrix}
a_{11} & a_{12} & \ldots & a_{1n}\\
a_{21} & a_{22} & \ldots & a_{2n}\\
\vdots & \vdots & \ddots & \vdots\\
a_{m1} & a_{m2} & \ldots & a_{mn}
\end{pmatrix}$ \\
\\
\[\begin{text}
if
\end{text}
\quad
\begin{textbf}
v
\end{textbf}
 = \begin{pmatrix}
v_{1}, & \ldots, & v_{n} 
\end{pmatrix}
\quad
\begin{text}
then
\end{text}
\quad
\begin{textbf}
v^{t}
\end{textbf}
= \begin{pmatrix}
v_1\\
\vdots\\
v_n
\end{pmatrix}\]

\end{document}