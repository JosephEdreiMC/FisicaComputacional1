\documentclass[letterpaper,twocolumn,10pt,draft]{article}
\usepackage[utf8]{inputenc}
\usepackage{graphicx}
\usepackage{color}
\title{Homework 3}
\author{Joseph Edrei Moreno Cruz~\\ 
F\'isica Computacional 1} 
\date{September 26, 2017}
\begin{document}
\maketitle

\textbf{Introducci\'on}

El uso de referencias es algo com\'un en la preparac\'ion de trabajos escritos en casi todas las disciplinas donde se genere conocimiento. En particular, en ingenier\'ia y ciencias se utilizan en forma frecuente para ejemplificar, proveer soporte a afirmaciones y reconocer la autor\'ia de ideas de otras personas en
nuestro documento.\\

\includegraphics[draft=false,width=150pt]{example-image-b}\\

El estilo de las referencias debe ser consistente y, en lo posible, seguir las convenciones de la disciplina en que uno est\'a presentando el trabajo. En ingenier\'ia, y en particular en ingenier\'ia el\'ectrica, el estilo IEEE ha ganado predominancia durante las \'ultimas cuatro d\'ecadas, y se ha convertido en el estilo por defecto de las referencias.
\vspace*{2\baselineskip}

Este documento es un mal ejemplo, pues no estoy colocando referencias. Sin embargo, en ejemplos posteriores veremos c\'omo colocar las referencias. Las referencias utilizadas en el cuerpo de una presentaci\'on escrita (art\'iculo de revista, conferencia, reporte t\'ecnico, memoria) deben agruparse en una lista que se adjunta al final del mismo documento. La lista de referencias debe incluir todas las fuentes mencionadas en el cuerpo del documento, y no m\'as.
\\

\textbf{Cambiando de color al texto} \par
Esta parte del texto muestra dos diferentes ejemplos en los que se indica c\'omo utilizar el paquete color para cambiar el color de elementos en \LaTeX .\\
 Puedes utilizarlo al momento de agregar listas:
\begin{itemize}
\color{blue}
\item Primer elemento
\item Segundo elemento
\end{itemize}

\textcolor{red}{O puedes f\'acilmente colocar un color diferente a una o varias palabras a lo largo del p\'arrafo que escribas.}

\end{document}